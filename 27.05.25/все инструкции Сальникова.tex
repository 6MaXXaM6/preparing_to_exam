\documentclass[a4paper,10pt]{article}
\usepackage[utf8]{inputenc}
\usepackage[english,russian]{babel}
\usepackage[top=2cm,bottom=3cm,left=0.5cm,right=2cm,nohead]{geometry}
\usepackage{multicol}
\usepackage{tabularx}


\begin{document}
\section*{пару слов по арифметике}
Есть у нас инструкция загрузки эффективного адреса очевидно что \\
lea ebx, [eax]; [eax] - указатель на память с адресом eax, поэтому ebx:=eax \\
но вспомним что адресные выражения могут быть разные и включать несколько регистров, тогда мы получаем инструкцию что может быстро выполнить арифметику за 1 такт сделав и умножение и сложение и не выставив флаги \\
пример: \\
lea eax, [eax+2*eax] \\ умножили eax на 3 так, как помним адресные выражения то можем умножить на степень двойки один из регистров прибавить убавить константу
для себя умножить ebx на 5, сложите eax и ebx и прибавьте 4\\
\section*{Инструкции}
зная вас вы возможно так и не поняли что за анонимные метки: \\
при компиляции все анонимные метки заменяются по аналогии с local метками в макросах на ??0000 в 16 сс
\begin{multicols}{2}
    \begin{verbatim}
        je @F; (Future)
            <код>
        @@:
    \end{verbatim}
    \columnbreak
    \begin{verbatim}
        @@: 
            <код>
        je @B; (Back)
    \end{verbatim}
\end{multicols}
\vspace{0.5cm}
инструкции по работе с флагами все без операндов \\
\begin{tabular}{|c|c|}
    \hline
    cld & DF:=0 строковые операции на увеличение адресов \\
    std & DF:=1 строковые операции на уменьшение адресов \\
    clc & CF:=0 \\
    stc & CF:=1 \\
    cmc & CF:=not(CF) \\
    cli & IF:=0 Interrupt Flag замаскировать прерывания (кроме №2) прерывания будут игнорироваться \\
    sti & IF:=1 Interrupt Flag вернуть обычное поведение с прерываниями \\
    lahf & загрузить в \textbf{ah} арифметические флаги AH := EFLAGS(SF,ZF,0,AF,0,PF,1,CF)\\
    sahf & загрузить из \textbf{ah} арифметические флаги EFLAGS(SF,ZF,0,AF,0,PF,1,CF) := AH\\
    pushfd & загрузить в стек 32 рязрядный EFLAGS \\
    popfd & забрать 32 разрядный EFLAGS \\
    \hline
\end{tabular}
\vspace{1cm} \\
инструкции перехода \\
напомню что они могут быть
\begin{enumerate}
    \item короткий rel8
    \item длинный rel16/rel16
\end{enumerate}
rel - непосредственный \\
напомню что также можно сделать \\
jmp \$+8 (переход на 8 байт вперёд по адресу) - если что компилятор сам так сделает если посчитает что это возможно и будет быстрее \\
и так тоже можно: jmp eax \\
\begin{tabular}{|c|c|}
    \hline
        инструкция&флаги \\
        je, jz & ZF=1 \\
        jne, jnz & ZF=0 \\
        jg, jnle & SF=OF and ZF =0 \\
        jge, jnl & SF=OF \\
        jl, jnge & SF<>OF \\
        jle, jng & SF<>OF or ZF=1 \\
        ja, jnbe & CF=0 and ZF=0 \\
        jae, jnb, jnc & CF=0 \\
        jb, jnae, jc & CF=1 \\
        jbe, jna & CF=1 or ZF=1 \\
        js & SF =1 \\
        jns & SF =0 \\
        jo & OF =1 \\
        jno & OF =0 \\
        jcxz & CX = 0 \\
        jecxz & ECX = 0 \\
    \hline
\end{tabular} \\
напомню мнемонику: e - equal - для любых, g -greater больше, l - less меньше- для знаковых, a-above (выше) b - below (ниже) - для беззнаковых, n - not, cx ecx это и есть регистры) \\
loop - сначала уменьшит ecx потом выполнит переход если <> 0 то есть ecx := 15 тогда 15 раз код выполнится\par
\noindent
условная пересылка инструкции \textit{c}\textbf{mov}\textit{cc} op1, op2\\
op1 - register 16 бит или register 32 бит, op2 - register/memory 16 бит или register/memory 32 бит соответственно \\
сувать константу ошибка! все условия проверки аналогичны инструкциям условного перехода, отсутсвуют лишь условная пересылка по регистру ecx и cx пример\\
\begin{verbatim}
    .code
        cmp bx, cx; напомню нельзя память память и константу первым параметром
        cmovb bx, cx; положит в bx max(bx, cx) беззнаково
\end{verbatim}
следующие инструкции как по мне уже очевидны добавлю только то что размеры одинаковы и единственное что запрещено 1 -операнд - константа и оба операнда память:
\begin{enumerate}
    \item[] mov
    \item[] xchg
    \item[] add
    \item[] sub
    \item[] inc - увеличить на 1 влияет на OF SF ZF но не CF r/m
    \item[] dec - уменьшить на 1 влияет на OF SF ZF но не CF r/m
    \item[] neg - берёт отрицание числа
    \item[] adc - op1 := op1+(CF+op2) - для сложения 64 битных чисел допустим
    \item[] sbb - op1 := op1-(CF+op2) - для вычитания
    \item[] lea - взять адрес
    \item[] mul AX=AL*r/m8 (8 битное)
    \item[] mul DX:AX= AX*r/m16 (16 битное)
    \item[] mul EDX:EAX= EAX*r/m32 (32 битное)
    \item[] imul - знаковое умножение
    \item[] div 8 битное беззнаковое: AL := AX div r/m8 AL := AX mod r/m8 
    \item[] div 16 битное беззнаковое: AX := DX:AX div r/m16 DX := DX:AX mod r/m16 
    \item[] div 32 битное беззнаковое: EAX := EDX:EAX div r/m32 EDX := EDX:EAX mod r/m32 
    \item[] idiv знаковое деление
    \item[] movsx - знаковое расширение
    \item[] movzx - беззнаковое расширение
\end{enumerate}
Инструкции знакового расширения применяются чаще всего при знаковом делении!!
\begin{enumerate}
    \item cbw - convert byte to word AL -> AX знаково
    \item cwd - convert word to double word AX -> DX:AX
    \item cdq - convert double to quadro EAX -> EDX:EAX
    \item cwde - convert word to double EAX AX -> EAX
\end{enumerate}
\vspace{0.5cm} 
(скажу разве что первым параметром у всех операций сдвигов и логики может быть память) \\
Логические операции : and, or, xor, not \\
\textbf{2 операнд сдвигов - imm8 (константа) или регистр CL!!} \\
Сдвиги: shr, shl, sal, sar - \textbf{SH}ift \textbf{R}ight, \textbf{S}hift \textbf{A}rithmetic \textbf{L}eft (ну мнемоника надеюсь ясна) \\
shl и sal одинаково действую с заполнением нулями битов справа \\
shr заполняет слева нулями, sal заполняет слева или справо так чтобы сохранить знак \\
бит что был сдвинут отправляется в флаг CF \\
Циклические сдвиги rol,ror, rcl, rcr - циклические сдвиги\\
rol и ror цикл внутри операнда но сдвинутый бит оказывается в CF \\
rcl и rcr цикл внутри операнда+CF те он на примере rcl ax, 1: сделает сдвиг влево и в младший (правый разряд) положит содержие CF при этом в после инструкции в CF окажется старший бит что был сдвинут \\
\vspace{0.5cm} \\
векторные инструкции я где то писал но повторюсь: \\
\textbf{<имя инструкции>[s|p][s|d]} - первый выбор между s  (scalar) и p (packed) означает будем ли мы обращаться с 1 элементом в xmm1 как правило элементом что в младшей части регистра или будем обращаться ко всем сразу второй выбор s (single precision) и d (double precision) за точность 32 битные числа или 64 битные \\
\begin{multicols}{2}
    \noindent
    addss xmm, xmm/mem32 - сложить \\
    addsd xmm, xmm/mem64  \\
    subss xmm, xmm/mem32 - вычесть \\
    subsd xmm, xmm/mem64 \\
    mulss xmm, xmm/mem32 - умножить \\
    mulsd xmm, xmm/mem64 \\
    divss xmm, xmm/mem32 - делить \\
    divsd xmm, xmm/mem64 \\
    sqrtss xmm, xmm/mem32 - корень \\
    sqrtsd xmm, xmm/mem64 \\
\end{multicols}
по инструкциями с packed также можно брать из памяти но память должна быть выравнена то есть адрес начала m128 в 16сс в конце имеет 0 те если из стека использовать то сохраняем стек и выполняем and esp, FFFFFFF0h и потом уже пушим что хоти складывать если данные описаны в .data то используем align 16: \\
\begin{multicols}{2}
    \noindent
    addps xmm, xmm/mem128 - сложить \\
    addpd xmm, xmm/mem128  \\
    subps xmm, xmm/mem128 - вычесть \\
    subpd xmm, xmm/mem128 \\
    mulps xmm, xmm/mem128 - умножить \\
    mulpd xmm, xmm/mem128 \\
    divps xmm, xmm/mem128 - делить \\
    divpd xmm, xmm/mem128 \\
    sqrtps xmm, xmm/mem128 - корень \\
    sqrtpd xmm, xmm/mem128 \\
\end{multicols}

Подробно про сравнения и cvt в ответах 25 числа напомню что в xmm не хранятся целочисленные их хранят в mmx! и что packed (cvtps2pi) преобразует лишь 2 числа (в младших битах) чаще всего использовать придётся cvtsi2ss и cvtss2si integer хранится в r/m32 число float в m32/xmm в память класть напрямую нельзя и иллюcтрация shufps - очень важная инструкция если будет практическая задача\\
fld - положить на вершину стека fpu (st0) операнд - память \\
fstp - достать из вершины (st0) - модно преобразовывать в real10 так\\
movss, movsd, movups, movupd, movapd, movapd - mem|xmm, mem|xmm (память в память нельзя) последние 2 инструкции требует ровнения по 16, также как и с packed операциями для арифметики \\
\section*{Строковые команды}
последняя буква - b|w|d - byte|word|double \\
как уже писали раннее DF=0 смотрим строку вперёд, DF=1 назад \\
std, cld - установить DF=1 и DF=0 соответственно \\
Мнемоника использования регистров: \\
ESI - Source \\
EDI - Destination \\
EAX|AX|AL - accumulator \\
ECX - counter \\
\vspace{0.5cm}
cmps[b|w|d] - сравнить esi c edi (псевдокод: cmp [esi],[edi])\\
movs[b|w|d] - переслать из esi в edi\\
scas[b|w|d] - Аккумулятор сравнивается с edi (псевдокод: cmp EAX,[edi]) \\
lods[b|w|d] - загрузить в Аккумулятор из esi \\
stos[b|w|d] - Загрузить в edi Аккумулятор \\
rep - повторять до ecx = 0 \\
repe - повторять пока равны и ecx <> 0 \\
repne - повторять пока не равны и ecx <> 0 \\
читайте лучше методичку по строковым она здесь лежит) там подробнее
\end{document}