\documentclass[a4paper,10pt]{article}
\usepackage[utf8]{inputenc}
\usepackage[english,russian]{babel}
\usepackage[top=2cm,bottom=3cm,left=0.5cm,right=2cm,nohead]{geometry}
\usepackage{multicol}

\begin{document}
\section*{Файл Теория}
Специальные регистры и их роль:
\textit{eip} -- cчётчик адреса, \textit{cs}, \textit{ds}, \textit{es}, \textit{ss}, \textit{fs}, \textit{gs} -- сегментные регистры,
регистры \textit{gdtr}, \textit{ldtr}, \textit{idtr} -- указатели на таблицы, \textit{tr} -- регистр задачи. \\
\vspace{0.5cm} \\
eip - регистр что хранит адрес на текующую исполняемую команду, единственный способ достать значение через call
\vspace{0.5cm} \\
сегментные регистры (дексриптор - описатель):
В архитектуре x86 сегментные регистры (CS, DS, ES, SS, FS, GS) используются для адресации памяти соблюдения уровней привелегий и защиты сегментов и для адресации. Они хранят селекторы сегментов — указатели на структуры данных в памяти (глобальную или локальную таблицу дескрипторов), которые описывают характеристики сегмента (базовый адрес, размер, права доступа и т.д.).
Селектором называется 16-битовое значение, первые 13 бит из них являются индексом дескриптора в одной из двух таблиц дескрипторов (таблица – это массив дескрипторов). Ещё один бит-индикатор (Table-Indicator) указывает, в какой таблице дескрипторов (глобальной или локальной gdt ldt)
находится дескриптор. Последние два бита задают уровень привилегий данного селектора. Дескриптор описывает какой-либо важный объект системы (сегмент, шлюз
прерывания, задачу (процесс), локальную таблицу дескрипторов и т.д.). \\
Дескрипторы сегментов хранятся в специальных таблицах, каждый такой дескриптор описывает
конкретный сегмент и содержит:
\begin{enumerate}
    \item адрес начала этого сегмента в оперативной памяти,
    \item максимальную длину (предел) сегмента,
    \item права доступа (разрешено ли чтение, запись и выполнение команд),
    \item уровень необходимых привилегий для работы с сегментом
    \item и некоторые другие атрибуты.
\end{enumerate}
у каждого сегментоного регистра есть теневой регистр для хранения дескриптора, для быстроты, но с ними работать напрямую нельзя, они теневые, дескрпитор сегмента можно достать используя индекс (первые 13 бит) и указатель на необходимую табличку ldtr gdtr (указатели на локальную и глобальную таблицы дескрпиторов) \\
\begin{enumerate}
    \item CS (Code Segment / Сегмент кода): Содержит селектор сегмента, в котором находятся исполняемые инструкции программы. Указатель инструкций IP/EIP/RIP содержит смещение внутри этого сегмента. запись запрещена для пользователя. Сегмент изменяется инструкциями перехода (дальние CALL, JMP, RET, прерывания, исключения).
    \item DS (Data Segment / Сегмент данных): Содержит селектор сегмента по умолчанию для большинства операций с данными (чтение/запись). Используется, когда в инструкции явно не указан другой сегментный регистр.
    \item ES (Extra Segment / Дополнительный сегмент): Исторически был "дополнительным" сегментом данных. Часто используется как регистр-получатель в строковых операциях (MOVS, STOS, CMPS и т.д.) вместе с DS (источник) или для доступа к специальным структурам данных.
    \item SS (Stack Segment / Сегмент стека):  Содержит селектор сегмента, в котором находится стек программы. Указатель стека ESP содержит смещение (вершину стека) внутри этого сегмента. Разрешает чтение/запись, но не исполнение. Изменяется при переключении задач/потоков.
    \item FS (Extra Segment 2 / Дополнительный сегмент 2): указывает на  (Thread Information Block), содержащий указатель на структуру TEB (Thread Environment Block) с критической информацией о текущем потоке (исключения, стек, локальное хранилище потока (TLS), ID потока, PID процесса и т.д.).
    \item GS (Extra Segment 3 / Дополнительный сегмент 3): GS указывает на TEB текущего потока (в 64-битном режиме роль FS и GS меняется по сравнению с 32-битным).
    \item TR (Task register): регистр задачи содержит селектор сегмента TSS (task status segment) используются осью для переключения между счётом задач
\end{enumerate}
В современном ПО (особенно в 32/64-битном защищенном режиме) DS, ES, SS часто указывают на один и тот дескриптор с полными правами на данные. \\
FS GS тесно связаны с курсом ОС достаточно просто сдополнительный сегментный регистр \\
TR - тоже про ОСИ но стоит помнить мы запускаем программу она порождает задачи задачи - потоки вот в TR - селектор сегмента в котором сохраняется все текущие данные чтобы позднее восстановить, после перехода.
Уровень привилегий тоже не проблема курса архитектура ЭВМ мы никчёмные у нас 3=CPL мы в ядро не лезем\\
\vspace{0.5cm} \\
Каждый номер прерывания определяет для процессора так называемый дескриптор (описатель) процедуры-обработчика прерывания с данным номером. Все такие дескрипторы (каждый длиной по 8 байт) располагаются в специальной таблице (массиве) дескрипторов прерываний с
именем IDT (Interrupt Descriptor Table), максимальная длина этой таблицы 2048 байт. По
существу, номер прерывания является индексом в массиве IDT. На начало IDT указывает системный
регистр IDTR, так что эта таблица может располагаться в любом месте памяти, а не обязательно с
нулевого адреса, как было в младших моделях первого и второго поколений процессоров семейства
Intel. Для многоядерных ЭВМ у каждого процессорного ядра свой регистр IDTR и, соответственно,
своя IDT.
\end{document}