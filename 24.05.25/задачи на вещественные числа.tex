\documentclass[a4paper,10pt]{article}
\usepackage[utf8]{inputenc}
\usepackage[english,russian]{babel}
\usepackage[top=2cm,bottom=3cm,left=2cm,right=2cm,nohead]{geometry}

\begin{document}

Считайте что это и есть тест

\section*{задачи на вещественные числа (не затрагивая прогу)}
\begin{enumerate}
    \item перечислите все необычные состояния вещественные числа float типа, когда они достигаются. на примере real32, real64
    \item заполните табличку числом разрядов \\
    \begin{tabular}{|c|c|c|c|c|c|}
        \hline
        стандарт& размер & бит знака & порядок & мантисса & bias \\
        \hline
        half precision & & & & \\
        \hline
        single precision & & & & \\
        \hline
        double precision & & & & \\
        \hline
        quad precision & & & & \\
        \hline
        extended precision & & & & \\
        \hline
    \end{tabular}
    \item как громкий NaN переделать в тихий NaN
    \item что выдаст sqrt(-1.0f) ;;C код
    \item В каких случаях float может считаться денормализованным
    \item переведите в число единичной точности\\
    +0 \\ 
    86.125 \\
    196.75 \\
    1/3 \\
    -0 \\
    $20*2^{-128}$ \\
    \item сложите два числа half precision: \\
    890.5 + 10.5625\\
    прокомментируйте все сдвиги поэтапно \\
    \item решите уравнение считая что числ single precision округление - к чётному
    98.3125+X=98.3125\\
    решите это же самое уравнение уже в half precision, сравните результат \\
    \item достижение наибольшего и наименьшего значения float \\
\end{enumerate}

\section*{задачи на ассемблере}
\begin{enumerate}
    \item Укажите разрядность XMM регистра, сколько их
    \item Согласено соглашению (cdecl, stdcall) как передаётся вещественное число
    \item реализовать сложение, вычитание чисел (дополнительно умножение, деление) одинарной и двойной сложности используя SSE2
    \item Сложить NaN c любым другим числом
    \item Сравнить поведение qNaN и sNaN
    \item Приведите операции что в результате дадут (real4 (single precision)):
    \begin{enumerate}
        \item +inf
        \item -inf
        \item NaN
        \item +0
        \item -0
    \end{enumerate}
\end{enumerate}
\end{document}