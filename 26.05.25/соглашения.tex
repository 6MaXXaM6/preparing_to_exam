\documentclass[a4paper,10pt]{article}
\usepackage[utf8]{inputenc}
\usepackage[english,russian]{babel}
\usepackage[top=2cm,bottom=3cm,left=0.5cm,right=2cm,nohead]{geometry}
\usepackage{multicol}

\begin{document}
\section*{соглашения Stdсall}
\begin{enumerate}
    \item можно изменять eax, edx, ecx - остальные менять нельзя
    \item стек чистит процедура
    \item параметры принимает только через стек первый параметр ближе к началу памяти поэтому push его последний(справо налево)
    \item результат возвращается al, ax, eax, edx:eax, st0
    \item по наименованиям:
    \begin{enumerate}
        \item внешние переменные (данные) начинаются c \_
        \item метки с \_
        \item процедуры начинаются с \_ заканчиваются @N - N чиcло байтов параметров кратно 4
    \end{enumerate}
\end{enumerate}
\section*{соглашения cdecl}
\begin{enumerate}
    \item можно изменять eax, edx, ecx - остальные менять нельзя
    \item стек чистит тот кто вызвал
    \item параметры принимает только через стек первый параметр ближе к началу памяти поэтому push его последний
    \item результат возвращается al, ax, eax, edx:eax, st0
    \item по наименованиям:
    \begin{enumerate}
        \item Символ подчеркивания (\_) префиксируется в имена, за исключением случаев, когда экспортируются \_\_cdecl функции, использующие компоновку C. \\
        мы на ассемблере и паскале считайте всегда просто нижнее подчёркивание должно быть
        \item .model, c ; заставит ассемблер использовать cdecl для имён
    \end{enumerate}
\end{enumerate}
\section*{соглашения fastcall}
\begin{enumerate}
    \item можно изменять eax, edx, ecx - остальные менять нельзя
    \item стек чистит процедура
    \item слева направо первые параметры что возможно засунет в ecx, edx другие параметры принимает через стек первый параметр ближе к началу памяти поэтому push его последний(справо налево)
    \item результат возвращается al, ax, eax, edx:eax, st0
    \item по наименованиям:
    \begin{enumerate}
        \item все имена начинаются с @
        \item после имени для процедур указание как в stdcall @N - число байтов с учётом регистров
    \end{enumerate}
\end{enumerate}
\section*{соглашения pascal}
\begin{enumerate}
    \item можно изменять eax, edx, ecx - остальные менять нельзя
    \item стек чистит процедура 
    \item параметры принимает только через стек первый параметр ближе к концу сегмента поэтому push его первый(слева направо)
    \item результат возвращается al, ax, eax, edx:eax, st0
    \item по наименованиям:
    \begin{enumerate}
        \item всё капсом
    \end{enumerate}
\end{enumerate}
\section*{соглашения pascal (register) практическая инфа}
\begin{enumerate}
    \item читай pascal
    \item стек чистит вызывающий
    \item параметры принимает через регистры слева направо для регистро eax, edx, ecx стек тоже слева направо первый аргумент первым и pushится
    \item результат возвращается al, ax, eax, edx:eax, st0
    \item по наименованиям Pascal переименует при extern только cdecl лучше каждый раз самим писать extern name <имя>
\end{enumerate}
\section*{практическая инфа fpc-masm}
\begin{enumerate}
    \item если будет задача на использовать функцию из fpc в masm то по умолчанию (если не указать в pascal у процедуры или функции - register/stdcall/cdecl/pascal) будет register соглашение
    \item pascal сам переименует на венгерский манер все названия (я спросил это запоминать не надо если что укажут как назвать) там длинные имена
    \item Если задача стоит наоборот в fpc использовать masm то следует указать название так как fpc не переименует (только cdecl переименует я проверял)
    \item Masm же можно заставить работать на соглашениях stdcall cdecl (c) и pascal - он всё переименует при компиляции согласно стандарту и раскроет макросы: invoke согласно соглашению глобально
\end{enumerate}
перепишите функцию c stdcall на cdecl, fastcall, pascal и подпишите как следует назвать функцию согласно соглашению если в .model не будет прописано соглашение
\begin{verbatim}
    .data 
        myr4n1 real4 8.5
    ; функция что будет рассмотрена
    sum proc
        push ebp; создание стекового кадра
        mov ebp, esp
        sub esp, 4; место под результат сложения
    
        cvtsi2ss xmm0, dword ptr [ebp+12]
        addss xmm0, dword ptr [ebp+8]
    
        movss [esp], xmm0
        fld dword ptr [esp]
    
        mov esp, ebp
        pop ebp
        ret 8
    sum endp
    
    Start:
        push 7
        push myr4n1; 8.5
    
        call sum
    
        fstp z4; просто место в памяти
        outf z4; распечатка 15.5
    
        exit
    end Start
\end{verbatim}
\end{document}