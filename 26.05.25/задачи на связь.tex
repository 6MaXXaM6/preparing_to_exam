\documentclass[a4paper,10pt]{article}
\usepackage[utf8]{inputenc}
\usepackage[english,russian]{babel}
\usepackage[top=2cm,bottom=3cm,left=0.5cm,right=2cm,nohead]{geometry}
\usepackage{multicol}

\begin{document}
\section*{задачи на связь fpc и masm}
большая часть задач просто на пощупать не переживайте если ничего не понимаете
\begin{enumerate}
    \item для чего существует ebp?
    \item опишите что делает ret
    \item опишите что делает call
    \item Вспомните сделать процедуру из fpc доступной изве (подсказка надо вспомнить как писать модули)
    \item за что отвечает ключевое слово asm внутри функции \\
    изучите  файл test\_inline\_assembly (можно написать свою ассемблерную вставку)
    \item изучить access\_to\_rand\_example - в этом примере функция рандома берётся из fpc (компилятор fpc сам подставляет из библиотек) и используется внутри asm из которого уже экспортируются функции в головной модуль который получается после компиляции prant\_array 
    \item для чего существует директива {\$L <имя файла>}
    \item откройте файл в директории firstupr firstupr.pas (необходимо написать процедуру внешнюю для fpc) (решение уже лежит на момент пуша если затрудняетесь)
    \item задание аналогично см secondupr.pas только попрошу написать во всех соглашениях что мы проходим кроме fastcall вместо него register так как fpc не работает с ним: cdecl, stdcall, pascal, register
\end{enumerate}
\end{document}