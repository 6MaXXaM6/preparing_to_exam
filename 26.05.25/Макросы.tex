\documentclass[a4paper,10pt]{article}
\usepackage[utf8]{inputenc}
\usepackage[english,russian]{babel}
\usepackage[top=2cm,bottom=3cm,left=1cm,right=2cm,nohead]{geometry}
\usepackage{multicol}

\begin{document}
\section*{теория}
\begin{enumerate}
    \item Что может быть фактическим параметром макрокоманды \par
    любой текст (в том, числе пустой), сбалансированный по кавычкам и угловым скобкам, причём
    этот текст не должен содержать запятых и точек с запятой вне вышеуказанных кавычек и скобок.
    \item что происходит с параметрами при макровызове \par
    при макровызове макропроцессор (=макрогенератор) находит макроопределение и подставляет в
    макрос фактические параметры вместо формальных, генерируя, таким образом, макрорасширение.
    При ответе на этот пункт нужна ещё такая добавка: при подстановке фактического параметра
    производится выполнение макрооператоров (\&,<>,\%,!)
    \item После передачи параметров начинается просмотр тела макроопределения и поиск в нём предложений, содержащих макросредства, например, макрокоманд. Все предложения в макроопределении,
    содержащие макросредства, обрабатываются Макропроцессором так, что в результате получается
    набор предложений на «чистом» языке Ассемблера (уже без макросредств), такой набор предложений называется \textbf{макрорасширением} Последним шагом в обработке макрокоманды
    является подстановка полученного макрорасширения на место макрокоманды, это действие называется \textbf{макроподстановкой} 
    \item \%: посчитает макровыражение пример:
\begin{multicols}{2}
    \begin{verbatim}
.code
    myMacro macro op1:req
        echo op1
    endm
Start:
    M equ (3*5+7)
    N equ mystring
    myMacro M
    myMacro %M
    myMacro N
    myMacro %N
    myMacro %7+7
    myMacro 7+7
    exit
end Start
    \end{verbatim}
    \columnbreak
вывод: \\
M \\
22 \\
N \\
mystring \\
14 \\
7+7 \\
\end{multicols}
\item ! - символ экранирования
\begin{multicols}{2}
\begin{verbatim}
.code
    myMacro macro op1:req
        echo op1
    endm
Start:
    M equ 7*6
    myMacro HelloWorld%M
    myMacro HelloWorld!%M
    myMacro HelloWorld!!!%M
    myMacro Helloworld!!
    myMacro !Helloworld
    myMacro Helloworld!
    exit
end Start
\end{verbatim}
    \columnbreak
вывод \\
HelloWorld42 \\
HelloWorld\%M \\
HelloWorld!\%M \\
Helloworld! \\
Helloworld \\
test.asm(17) : error A2038: missing operand for macro operator \\
 myMacro(1): Macro Called From \\
  test.asm(17): Main Line Code \\
Helloworld! \\
\end{multicols}
отметим что \% и ! поставленные в конце выражения выдадут ошибку, необходимо экранировать
\item Если фактический параметр заключён в угловые скобки (<>), то эти скобки считаются макрооператорами выделения, они заставляют рассматривать заключённый в них текст как единое целое, даже если в нём встречаются служебные символы. Обработка макровыделения заключается в том, что внешние парные угловые скобки отбрасываются при передаче фактического параметра на место формального. Обратите внимание, что отбрасывается только
одна (внешняя) пара угловых скобок, если внутри фактического параметра есть ещё угловые
скобки, то они сохраняются.
\begin{multicols}{2}
\begin{verbatim}
.code
    myMacro macro op1:req, op2:=<default>
        echo op1
        echo op2
    endm
Start:
    SEMICOLON_CHAR TEXTEQU <;>
    M equ 7*6
    myMacro <HelloWorld%M, I`m>
    myMacro <HelloWorld%M ; I`m> ; внутри комментарий
    myMacro HelloWorld%M, I`m
    exit
end Start
\end{verbatim}
    \columnbreak
вывод \\
HelloWorld42, I`m \\
default \\
HelloWorld42 \\
default \\
HelloWorld42 \\
I`m \\
\end{multicols}

\item \& - разделитель лексемы! важно могло слоиться неправильное впечатление что это указатель или разыменователь но это не так то что необходимо подставить надо выделять! пример \\
\begin{multicols}{2}
\begin{verbatim}
M1 macro x
    for i,<1,2,3>
        x&&i db i
    endm
endm
.data
    M1 myvar
    ; сгенерит myvar1 db 1 и другие
\end{verbatim}
с двух сторон отделить то что подставляешь за исключением конца и начала строки там только с одной стороны если у нас подстановка из одного имени то \& не нужен\\
\columnbreak
\begin{verbatim}
M1 macro x, ze
    for i,<1,2,3>
        x&U&ze&&i&q db '$i$'
    endm 
endm
.data
    M1 myvar, _
    ; сгенерит myvarU_1q db '1' и другие
\end{verbatim}
иной пример
\end{multicols}
что будет если \\
если записать x\&i то i не подставится: myvari \\
если записать x\&U\&ze\&\&i\&q но макропеременной ze нет:  myvarU\&ze\&1q \\
если записать x\&U\&ze\&i\&q: myvarU\&ze1q
\end{enumerate}
\newpage
табличка конструкций \\
\begin{tabular}{|c|c|}
    \hline
        type & размер одно элемента в байтах если константа то 0, \\
    \hline
        size & размер элементов до первой запятой  size real8 6 dup (?) = 48 \\
    \hline
        sizeof & размер элементов включая все перечисления через запятую \\
    \hline
        length & число элементов до запятой то есть вернёт 1 или то что перед dup  \\
    \hline
        lengthof & число всех элементов после метки \\
    \hline
\end{tabular} \\
директивы условной компиляции \\
\begin{tabular}{|c|c|}
    \hline
    ifb <op1>& (if blank) если параметр пустой то подстановку выполняем \\
    ifnb <op1>& (if not blank) \\
    ifdif <op1>, <op2>& (if difference) если разные то выполняем \\
    ifidn <op1>, <op2>& (if identical) если одинаковые то выполняем \\
    ifdifi <op1>, <op2>& (if difference ignore case) если разные регистронезависимо ('string' = 'StRiNg') то выполняем \\
    ifidni <op1>, <op2>& (if identical ignore case) если одинаковые регистронезависимо то выполняем \\
    \hline
\end{tabular} \par
Директива If выполняет подстановку согласно булевым значениям в макросах всё что не 0 это true, 0 = false \\
логические выражения \\
\begin{multicols}{3}
    \noindent
    EQ вместо =\\
    LE вместо $\le$ \\
    \columnbreak \\
    GE вместо $\ge$ \\
    LT вместо < \\
    \columnbreak \\
    GT вместо >\\
    NE вместо $\ne$ \\
\end{multicols}
любой обозначенный 'if' завершается endif внутри могут быть выражения: else и elseif[idn|idni|dif|...] \\
операции в макровыражениях и их приоритет:
\begin{enumerate}
    \item (), [], <>, length[of], size[of]
    \item . (точка)
    \item : (переопределение сегмента по умолчанию)
    \item  ptr, offset, seg, type, this
    \item high, low, highword, lowword (возвращает старшую и младшую половины от слова и от двух слов соответственно) 
    \item Одноместные (унарные) + и – (для корректности 6+-+7)
    \item *, / (в смысле div), mod, shl, shr
    \item бинарные + и –
    \item EQ, NE, LT, LE, GT, GE
    \item not
    \item and
    \item or, xor
    \item short, .type
\end{enumerate}
переходы внутри определния можно выполнять по макрометкам \\
перед их использованием необходимо их объявить в начале макроса прописав:
local L \\
после чего где то внутри макроса поставить метку: \\
:L (в отличии от меток asm : ставится перед), переход осуществляется по goto L\par
также директива local для Макропроцессора имеет следующий смысл. При каждом входе в макроопределение локальные имена, перечисленные в этой директиве, получают новые уникальные значения. Обычно Макропроцессор выполняет это совсем просто:
при первом входе в макроопределение заменяет локальное имя L на имя ??0001 \\
для преждевременного выхода из макроса используется .exitm \\
благодаря этой директиве можно заранее выйти из цикла или из всего макроса в случае ошибки при макропроцессировании \\
.err <сообщение> - породит ошибку при компиляции \par
\noindent
\textbf{opattr} операнд - возвращает 16 байт: \\
\begin{tabular}{|c|c|c|c|c|c|c|c|c|c|c|c|c|c|c|c|}
    \hline
    ? & ? & ? & ? & ? & ? & ? & ? & ? & ? & ? & ? & ? & ? & ? & ?  \\
    \hline
    15 & 14 & 13 & 12 & 11 & 10 & 9 & 8 & 7 & 6 & 5 & 4 & 3 & 2 & 1 & 0 \\
    \hline
\end{tabular}
\newpage
\noindent
;     Bit    Set If... \\
;     0      имя метки процеду \\ 
;     1      перемещаемая переменная (область данных)\\
;     2      константа \\
;     3      перемещаемый адрес \\
;     4      регистровое выражение \\
;     5      определён \\ 
;     6      операнд в стеке (usually a LOCAL variable or parameter) \\
;     7      имя в extrn label \\
;     8-10   Language type (0=no type) \\
что другие да фиг пойми вроде соглашения и чёт зарезервировано не нагуглил\\
первые 8 бит аналогичны .type \par
просто по циклам пример отрывок макроса проверить на регистр что можно менять по соглашению eax, edx, ecx  \\
\begin{verbatim}
local K
K=0; это var K: integer:=0
for reg,<eax, edx, ecx, ax, dx, cx, al, ah, dl, dh, cl, ch>
    ifidni <reg>,<X>
        K=1; это K:=1
        exitm; преждевременный выход из цикла
    endif
endm
\end{verbatim}
по циклу forc
\begin{verbatim}
local K
K=0; это var K: integer:=0
forc reg,<adc>
    ifidni <reg>,<e&reg&x>; проверка лишь на 4 байтные
        K=1; это K:=1
        exitm; преждевременный выход из цикла
    endif
endm
\end{verbatim}
цикл \textbf{while} выражение как в if - если что метки есть\\
цикл \textbf{repeat} N \\
\end{document}