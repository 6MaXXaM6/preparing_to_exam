\documentclass[a4paper,10pt]{article}
\usepackage[utf8]{inputenc}
\usepackage[english,russian]{babel}
\usepackage[top=1.5cm,bottom=2cm,left=1cm,right=1cm,nohead]{geometry}
\usepackage{multicol}
\usepackage{graphicx}
\usepackage{tabularx}

\begin{document}
\section*{Прогноз}
\textbf{№1} \\
перечислите принципы Фон неймана: \\
\vspace{0.3cm} \\
\textbf{№2} \\
сложите числа в half precision: \\
1915 + 11.5625
\vspace{0.3cm} \\
\textbf{№3}
Подчеркните синтаксически неправильные инструкции: \\
\begin{tabularx}{\textwidth}{|X|c|X|X|c|c|}
    \hline
    lea eax, [2*eax+4*edi]-7 & jmp eip & add byte ptr [eax], 0 & rol dword ptr [esi], bl & align 5\\
    \hline
\end{tabularx}
\vspace{0.2cm} \\
\textbf{№4} \\
напишите полную программу, в том числе секцию data, что найдёт длину наибольшего промежутка сумма элементов которого в массиве = 0 \textit{arrayword} длиной N (>0)  знаковых чисел, выведите результат, оформите в виде процедуры код соблюдая pascal массив передавать по ссылке можно менять главное выдать ответ\\
нарисуйте стек, в процессе исполнения процедуры (если используются локальные переменные их тоже) \\
\vspace{0.3cm} \\
\textbf{№5} \\
Нарисуйте связи что образует этот код (считать что по умолчанию процедуры приватные option proc:private) считать что masm ничего не переименует
\begin{multicols}{2}
a.asm
\begin{verbatim}
public openf
.data
    openf dd 0
    extrn programmer_id:tword
.code
Start:
    extrn drink_cofe:proc
    ...
end Start
\end{verbatim}
\columnbreak
b.asm
\begin{verbatim}
public programmer_id
.data
    programmer_id dt 0FFh
    extrn openf:dword, Ll:near
.code
drink_cofe proc public
...
drink_cofe endp
end
\end{verbatim}
\end{multicols}
Отметте головной модуль, слинкуется ли код? \\
\vspace{0.3cm} \\
\textbf{№6} \\
Описать в виде макроса NULL RS обнуление регистров общего назначения. Здесь RS — это строка вида <R1,R2,…,R$_k$>, R$_i$ — регистр общего назначения, k $\ge$ 0. Выписать макрорасширения для макрокоманд NULL <AL,BX,ESI> и NULL <> \\
\vspace{0.3cm} \\
\textbf{№7} \\
описать в виде процедуры um4x4 A, B - A и B ссылки на матрицы (16 написанных по порядку чисел строка за строкой), A:=A*B \\
\vspace{0.3cm} \\
\textbf{№8} \\
Что значит маскировать прерывание
\end{document}