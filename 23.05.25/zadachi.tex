\documentclass[a4paper,10pt]{article}
\usepackage[utf8]{inputenc}
\usepackage[english,russian]{babel}
\usepackage[top=2cm,bottom=3cm,left=2cm,right=2cm,nohead]{geometry}

\begin{document}

ВОЗМОЖНЫ ЕЩЁ задачи проверяйте на утро следующего дня)

\section*{задачи на УМ}
\begin{enumerate}
    \item расположить в порядке скорости выполнения арифмитечских операций УМ-0, УМ-1, УМ-2, УМ-3
    \item написать программу Даны 4 числа найти дисперсию (среднее квадратичное отклонение) УМ3 и УМ2
    \item Что такое ячейка, что такое адрес ячейки и самоидентифицирующаяся прогрмма
\end{enumerate}

\section*{задачи на числа}
\begin{enumerate}
    \item что происходит с знаковым числом если сделать побитовое отрицание
    \item Бородоченкова 3.5 a, б, 3.6 а, б
    \item Что будет neg eax, если в eax: -128, а если 127? 
    \item Ячеика памяти 8 бит укажите значение флагов ZF SF CF OF \\
    -120-115\\
    57+34\\
    160-20\\
    -127+2\\
    \item Реализуйте макрос чтобы чтобы пару EDX:ECX - целочисленный int сделать аналог neg для этой пары
\end{enumerate}
 
по фон Нейману вопросы найдёте в методичке Баулы

\end{document}