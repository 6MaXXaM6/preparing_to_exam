\documentclass[a4paper,10pt]{article}
\usepackage[utf8]{inputenc}
\usepackage[english,russian]{babel}
\usepackage[top=2cm,bottom=3cm,left=2cm,right=2cm,nohead]{geometry}
\usepackage{multicol}

\begin{document}
\section*{задачи по структуре программы}
\begin{enumerate}
    \item Чем отличается директива equ и '='
    \item в каком сегменте я могу написать equ
    \item Чем отличается и какие ошибки произойдут и на каком этапе
    \begin{multicols}{2}
\begin{verbatim}
.code
    N equ 8
    mov N, eax
\end{verbatim}
        \columnbreak
\begin{verbatim}
.const
    N dd 8
.code
    mov N, eax
\end{verbatim}
    \end{multicols}
    \item Опишите секцию .data?
    \item программа компилируется стандартно "ml /c /coff /Fl main.asm" будет ли ошибка?
\begin{verbatim}
.data?
    msg db 'Hello World', 0
\end{verbatim}
    \item опишите .stack, может ли эта секция встретиться в неглавном модуле?
    \item Бородоченкова 6.1 (на правильность написания адресных выражений)
    \item Бородоченкова 7.1 a) б) 7.2 а) 7.6 д) (7.6 ж)
    \item На листочке опишите структуру вашего холодильника (творчески)
    \item Выпишите весь вывод
\begin{verbatim}
include console.inc

metadata STRUC
    timestamp dq ?
    author db '6MaXXaM6',0
    testdata db 0FFh
metadata ENDS

.data
    typ db 5 dup ('6MaXXaM6'),0
    x metadata <12>, <23>

.code
start: 
    
    outu length metadata.author
    newline
    outu length typ
    newline
    outu lengthof metadata.author
    newline
    outu sizeof metadata.author
    newline
    outu size metadata.author
    newline
    outu sizeof metadata
    newline
    outu sizeof x
    newline
    outstrln offset x.author
    outu x.timestamp

    newline
    lea eax, x
    outu [eax+metadata.timestamp]

    exit
end start
\end{verbatim}
\end{enumerate}

\end{document}