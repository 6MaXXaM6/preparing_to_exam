\documentclass[a4paper,10pt]{article}
\usepackage[utf8]{inputenc}
\usepackage[english,russian]{babel}
\usepackage[top=2cm,bottom=3cm,left=2cm,right=2cm,nohead]{geometry}
\usepackage{multicol}


\begin{document}
\section*{задачи многомодульное программирование}
\begin{enumerate}
    
    \item ЕСЛИ вы не открывали пример многомодульного программирования откройте - можно на виртуалке всех команды в readme файлике после возвращайтесь!
    \item Синтаксис и Семантика public extrn
    \item Будет ли ошибка?
\begin{verbatim}
include console.inc

.code
myuniquestart: 
    outstrln 'HelloWorld'
    exit
end myuniquestart
\end{verbatim}
    \item Вопрос со звёздочкой: будет ли ошибка?
\begin{verbatim}
include console.inc

.code
start: 
    outstrln 'HelloWorld'
    exit
end
\end{verbatim}
    Если да то опишите как исправить (не приписывая после end start)
    \item ml /c /coff /Fl main.asm - стандартная компиляция object файла для нас (или через wine) поясните за каждый ключ \\
    \begin{tabular*}{15cm}{c|c}
        \hline
        /c &   \\
        \hline
        /coff & \\
        \hline
        /Fl & \\
        \hline
    \end{tabular*}
    \item опишите что случится с секциями .code разных модулей после линковки в 1 загружаемый файл.
    \item MyProc proto :byte,:dword,:byte - для чего существует возможность описывать прототип опишите ситуацию
    \item private для чего существует
    \item найдите все ошибки исправьте после линковки и запуска должен отработать next\_day взяв параметры как у прототипа
\begin{verbatim}
.686; доп модуль
.XMM

.model flat, stdcall
public next_day@12
.code
; рпринимает 3 параметра
next_day@12:
next_day proc private
push ebp
mov ebp, esp
...
pop ebp
ret 12
next_day endp
end
\end{verbatim}


\begin{verbatim}
.686; основной модуль
.XMM

.model flat

.data 
op1 db ?
op2 dd ?
op3 db ?
.code
Start:
    next_day proto :byte,:dword,:byte
    push op3
    push op2
    push op1
    call next_day
    ...
end Start
\end{verbatim}
    \item схематично стрелками как на лекции обрисуйте все выходы и входы сделайте вывод о линковке подпишите внешние имена в формате stdcall (указана опция: option proc:private)
\begin{multicols}{2}
\begin{verbatim}
;main.asm
public cringe, twop@4
.data
    extrn love:real4; 7F800000h))
    cringe dw ?
    
.code
twop@4:
twop proc
...
twop endp

onep proc public
...
onep endp
Start:
    ...
end Start
\end{verbatim}
\columnbreak
\begin{verbatim}
    ;module.asm
    public love
    
    .data
        love real4 ?
        extrn cringe:word
        extrn true:byte
        
    twop proto :dword
    .code
        extrn onep@0:proc
    ...
    end
\end{verbatim}
\end{multicols}
пролог в обоих случая Стандартный
\begin{verbatim}
.686

.model flat, stdcall
option proc:private
\end{verbatim}
    Если можно убрать ошибки совместимости вычеркиванием одной строки сделайте
\item дан листинг указать на месте подчёркиваний числа или буквы
\begin{multicols}{2}
\begin{verbatim}
00000000			.data
        extrn hapiness:byte
00000000  00000005 [		    garbage db 5 dup (?)
00
]
00000005 2A			    mydata db 42; 
\end{verbatim}
\columnbreak
\begin{verbatim}
    000000XX  88 25 ________ _	   mov hapiness, ah
    000000XX  C6 05 ________ _	   mov mydata, 8
    08
\end{verbatim}
\end{multicols}
\end{enumerate}

\end{document}