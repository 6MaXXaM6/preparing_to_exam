\documentclass[a4paper,10pt]{article}
\usepackage[utf8]{inputenc}
\usepackage[english,russian]{babel}
\usepackage[top=2cm,bottom=3cm,left=2cm,right=2cm,nohead]{geometry}

\begin{document}
этот файл прорешивается после написания мини-прака по счёту смешанного произведения причём попрошу загружать в xmm c помощью movaps и movups
\section*{векторные инструкции}
\begin{enumerate}
    \item укажите вывод
\begin{verbatim}
.data
    ar db ?
    align 16
    ar2 dd ?

.code
Start:
    
    lea eax, ar
    outhex eax; out: 004032EE
    newline

    lea eax, ar2 
    outhex eax; out: ?
    newline

    exit
end Start
\end{verbatim}
какой вывод был бы если align не было
    \item comiss comisd ucomiss ucomisd - в чём отличие приведите код и примеры данных на которых можно увидеть различие comiss и ucomiss - какие флаги они выставляют
    \item решить задачу first.asm дан массив: ar\_real4 элементов 20, сколько из них достаточно сложить по модулю (использовать andss или сравнение с 0 умножение на 1) если складывать их по порядку записи чтобы сумма была больше sample\_real4 (условия в директории)
    \item 
    \begin{enumerate}
        \item shufps xmm0, 00000110b - что произойдёт \\
        \begin{tabular}{|c|c|c|c|c|}
            \hline
             & xmm0[3] & xmm0[2] & xmm0[1] & xmm0[0] \\
             \hline
            было &  000D & 000C & 000B & 000A \\
            \hline
            стало &   &  &  &   \\
            \hline
        \end{tabular}
        \item shufps xmm0, 10110001b - что произойдёт \\
        \begin{tabular}{|c|c|c|c|c|}
            \hline
             & xmm0[3] & xmm0[2] & xmm0[1] & xmm0[0] \\
             \hline
            было &  000D & 000C & 000B & 000A \\
            \hline
            стало &   &  &  &   \\
            \hline
        \end{tabular}
        \item напишите 2 операнд shufps такой чтобы произошёл циклический сдвиг вправо
    \end{enumerate}

    \item расшифруйте:
    \begin{enumerate}
        \item cvtss2si 
        \item cvtsi2ss
        \item cvtsd2ss
        \item cvtps2pi
        \item cvtpd2ps 
    \end{enumerate}
    используя cvt посчитайте сумму всех целых частей чисел масисива из 3 задачи\\
    \item  какой ещё способ существует для конвертации чисел в том числе в extended? \\
    напите отрывок программы считая что \\
\begin{verbatim}
    z4 real4 ?
    z8 real8 ?
    z10 dt ?
\end{verbatim}
    конвертните z4 и положите в z8 и z10 не используя cvt
\end{enumerate}

\end{document}