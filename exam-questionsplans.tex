\documentclass[a4paper,10pt]{article}
\usepackage[utf8]{inputenc}
\usepackage[english,russian]{babel}
\usepackage[top=2cm,bottom=3cm,left=2cm,right=2cm,nohead]{geometry}

\begin{document}
\section*{Список вопросов, необходимо выучить}
\begin{enumerate}
    \item[23.05.25] первые 5 вопросов так и идём
    \begin{enumerate}
        \item[1] 
        Общая структура вычислительной машины. Шины, процессоры, оперативная память, внешние устройства.
        \item[2]
        Понятие архитектуры процессора и микроархитектуры. Основные компоненты.
        \item[3]
        Принципы архитектуры Джона фон Неймана. Цикл выполнения команды. минимально необходимый набор регистров.
        \item[4]
        Первые ЭВМ, первые устройства ввода и вывода. Учебные машины УМ-3, УМ-2, УМ-1, как примеры организации первых ЭВМ.
        \item[5]
        Машинное представление целых чисел: прямой код, дополнительный код. Выставление флагов знакового и беззнакового переполнения.
        Флаг знака и флаг нуля. Диапозоны знаковых и беззнаковых чисел. Получение обратного числа через побитовое отрицание и сложение.
    \end{enumerate} 
\end{enumerate}
\newpage
\section*{Список вопросов по курсу <<Архитектура ЭВМ и язык ассемблера>>}
\begin{enumerate}
    \item
        Общая структура вычислительной машины. Шины, процессоры, оперативная память, внешние устройства.
    \item
        Понятие архитектуры процессора и микроархитектуры. Основные компоненты.
    \item 
        Принципы архитектуры Джона фон Неймана. Цикл выполнения команды. минимально необходимый набор регистров.
        Число операндов в команде. CISC и RISC процессоры.
    \item
        Первые ЭВМ, первые устройства ввода и вывода. Учебные машины УМ-3, УМ-2, УМ-1, как примеры организации первых ЭВМ.
    \item
        Машинное представление целых чисел: прямой код, дополнительный код. Выставление флагов знакового и беззнакового переполнения.
        Флаг знака и флаг нуля. Диапозоны знаковых и беззнаковых чисел. Получение обратного числа через побитовое отрицание и сложение.
    \item
        Машинное представление чисел с плавающей точкой. Числа с плавающей точкой половинной, одинарной, двойной точности. Числа в 80 битном формате
        FPU блока x86 архитектуры. Специальные числа: бесконечности, неопределённого значения (NaN). Понятие машинного 0, проблемы округления
        чисел с плавающей точкой. Округление к ближайшему чётному.
    \item
        Общая структура программы на ассемблере в masm. Директивы, указание типа процессора и модели памяти.
            Секции \textit{.data}, \textit{.data?}, \textit{.code}, \textit{.stack}.
        Резервирование памяти в секции, конструкции:
            \textit{db}, \textit{dw}, \textit{dd}, \textit{dq}, \textit{dt}, \textit{real4},
            \textit{real8}. Огранизация повторения при резервировании памяти: конструкция \textit{dup}. 
            Резервирование строки.
    
            Описание констант.
    
    \item
       формат описания команд процессорв в masm. префикс, допустимые типы операндов. Косвенная адресация по одному и по двум регистрам.
       Конструкция \textit{offset}. Указание размера операнда при обращении в память. Метки в коде, в том числе работа с анонимными метками.
       Точка входа в программу (метка start) оформление головного ассемблерного модуля.
    
    \item
       Понятие компиляции и линковки ассемблерного кода. Модули на ассемблере и объектные модули.
       Указание внешних и внутренних имён в модулях, директивы \textit{public} и \textit{extern}.
       оформление не головного ассемблерного модуля.
    
    \item
       Регистры архитектуры x86-32. Назначение регистров. Регистры общего назначения, специальные регистры. Обращение к частям регистров.
       
       Регистр флагов и его поля.
            Инструкции очистки и установки флагов: \textbf{cld}, \textbf{std}, \textbf{clc}, \textbf{stc}, \textbf{cmc}, \textbf{cli},
            \textbf{sti}.
            Инструкция сохранения арифметических флагов в регистр \textit{ah}: \textbf{lahf}, восстановление \textbf{sahf}.
    
       Регистры блока FPU, векторные регистры SSE2.
    \item
        Специальные регистры и их роль:
            \textit{eip} -- cчётчик адреса, \textit{cs}, \textit{ds}, \textit{es}, \textit{ss}, \textit{fs}, \textit{gs} -- сегментные регистры,
            регистры \textit{gdtr}, \textit{ldtr}, \textit{idtr} -- указатели на таблицы, \textit{tr} -- регистр задачи.
    \item
        Инструкция пересылки \textbf{mov}. Семейство инструкций условной пересылки по состоянию флагов: \textbf{сmov}\textit{cc}.
        Инструкция обмена \textbf{xchg}.
    \item
        Арифметические инструкции: \textbf{add}, \textbf{sub}, \textbf{inc}, \textbf{dec}, \textbf{adc}, \textbf{sbb}, \textbf{neg}.
        Влияние данных инструкций на арифметические флаги.
    \item
        Инструкции умножения \textbf{mul}, \textbf{imul}, и деления \textbf{div}, \textbf{idiv} (только одноаргументный вариант).
    
            Инструкция взятия адреса -- \textbf{lea}. Использование для быстрого умножения на степени двойки (0,1,2,3) со сложением.
    
    \item
        Инструкции знакового и безнакового расширения типов: \textbf{movsx}, \textbf{movzx}.
    
        Инструкции знакового расширения: \textbf{cbw}, \textbf{cwd}, \textbf{cdq}, \textbf{cwde}.
    
    \item
        Побитовые операции: \textbf{and}, \textbf{or}, \textbf{xor}, \textbf{not}. Операции сдвигов: \textbf{shl}, \textbf{shr}, \textbf{sal}, 
        \textbf{sar}. Циклические сдвиги: \textbf{rcl}, \textbf{rcr}, \textbf{rol}, \textbf{ror}. 
        Влияние всех этих инструкций на арифметические флаги.
    \item    
        Инструкции для работы с числами с плавающей точкой в наборе инструкций SSE2.
        Скалярные операции: \textbf{addss}, \textbf{addsd}, \textbf{subss}, \textbf{subsd}, \textbf{mulss}, \textbf{mulsd}, 
                            \textbf{divss}, \textbf{divsd}.
        Векторные арифметические операции (упакованные данные) с суффиксом \textit{p}.
    
        Сравнение чисел с плавающей точкой: \textbf{comiss}, \textbf{comisd}, \textbf{ucomiss}, \textbf{ucomisd}. 
        Выставление флагов в регистре \textit{eflags} операциями сравнения.
    
        Инструкция горизонтального перемешивания частей вектороного регистра \textbf{shufps}.
    \item
        Семейство инструкций \textbf{cvt}\textit{XXX}\textbf{2}\textit{YYY} конвертации чисел с плавающей точкой. Использование FPU для
        конвертации: инструкция \textbf{fld} и \textbf{fstp}.
    \item
        Инструкции пересылки  числел с плавающей точкой в векторные регистры: \textbf{movss}, \textbf{movsd}, \textbf{movaps}, \textbf{movapd},
        \textbf{movups}, \textbf{movupd}.
        
        Инструкции пресылки в FPU и обратно: \textbf{fld} и \textbf{fstp}. 
    \item
        Инструкция сравнения \textbf{cmp} и побитового сравнения \textbf{test}.
    
        Инструкция перехода: \textbf{jmp}. Инструкция условного перехода \textbf{j}\textit{cc}. Переход по значению регистра \textit{ecx} --
        \textbf{jcxz}, \textbf{jecxz}.
    
        Инструкция цикла \textbf{loop}.
    
        Допустимые аргументы инструкций переходов.
    \item
        Строковые команды: \textbf{movs}\textit{l}, \textbf{cmps}\textit{l}, \textbf{scas}\textit{l}, \textbf{lods}\textit{l}, \textbf{stos}\textit{l}.
        Влияние флага направления DF, сменв направления.
    
        Префиксы повторения: \textbf{rep}, \textbf{repe}, \textbf{repne}.
    \item
        Работа со стеком. Роль регистра \textit{esp}. Инструкции \textbf{push} и \textbf{pop}. Допустимые аргументы инструкций \textbf{push} и \textbf{pop}.
        Инструкции сохранения и восстановления регистров \textbf{pushad} и \textbf{popad}. 
        Сохранение и восстановление регистра флагов: \textbf{pushfd}, \textbf{popfd}.
    \item
        Процедуры и функции. Инструкция \textbf{call}. Стековый кадр, роль регистра \textit{ebp}. Стандартный пролог и эпилог функции.
    \item
        Процедуры и функции, стандартные соглашения: \textit{stdcall}, \textit{fastcall}, \textit{cdecl}, \textit{pascal}. 
        Доступ к параметрам и передача параметров, организация локальных переменных, сохраняемые функцией/процедурой регистры,
        возвращение целочисленнго результата, возвращение результата как числа с плавающей точкой, очистка стека.
    
        Особенность передачи параметров по ссылке и по значению.
    
        Синтаксис оформления функции в masm.
    \item
        Организация структур и массивов. Конструкция \textit{equ}. Описание аналогов record языка Pascal конструкция \textit{struct}.
    \item
        Ассемблерные вставки в FreePascal. Организация совместной линковки FreePascal и объектных модулей. Указание внешних имён функций и процедур
        в FreePascal и стандартных соглашений.
    \item
        Макросредства masm. Директива \textit{macro}, параметры макроса: cо значениями по умолчанию, требуемые параметры. Указание параметров 
        в угловых скобках. Конструкции разрыва лексемы \textit{\&}. Конструкция экранирования символов: \textit{!}.
        Объявление макропеременных. Локальные метки и макропеременные. Конец макроса директива \textit{endm},
        Досрочный выход из макроса директива \textit{exitm}. Директива \textit{.err}. Печать информационного сообщения при компиляции  --- директива
        \textit{echo}.
    \item
        Конструкции определения типов и размеров в masm: \textit{type}, \textit{size}, \textit{sizeof}, \textit{length}, \textit{lengthof}.
        Определение атрибутов параметра: директива \textit{opatr}.
    \item
        Директивы условной компиляции: \textit{if}, \textit{ifb}, \textit{ifdiff}, \textit{ifdiffi},
                     \textit{ifidn}, \textit{ifidni}, \textit{else}, \textit{elseif}, \textit{endif}.
    \item
        Директивы циклов: \textit{for}, \textit{forc}
    \item
        Макросы ввода вывода, написанные преподавателями для проведения курса: inint, outstrln, outword, ...
    \item
        Микроархитектура процессора: теневые регистры, буфер команд, множественность арифметико-логических устройств, конвейерность устройств.
        КЭШ память, уровни КЭШ. Трансляция команд процессора во внутренние RISC команды.
    \item
        Конвейерность. Стадии конвейера, возможные причины простоя конвейера (образования <<пузырей>>), причины сброса конвейера.
        Механизм предсказания переходов и спекулятивного выполнения.
    \item
        Организация КЭШ памяти. Понятие строки/линии КЭШ. Попадание/промах в КЭШ. КЭШ прямого отображения, ассоциативный КЭШ.
        Сквозная и отложенная запись в КЭШ. Проблема синхронизации КЭШ для нескольких процессорных ядер/процессоров.
        Префикс \textbf{lock} у инструкций. Инструкции принудительной снхронизации: \textbf{clwb}, \textbf{clflush}, \textbf{sfence}, \textbf{lfence},
        \textbf{mfence}. Семейство инструкций записи в память минуя КЭШ \textbf{movnt}\textit{xx}.
        Инструкции подгрузки в КЭШ: \textbf{prefetch0}, \textbf{prefetch1}, \textbf{prefetch2}.
    
    \item
        Организация прерываний в архитектуре x86-32. Общие принципы организация обработки прерываний от внешних устройств. Внутренние прерывания, 
        Прерывания и исключения. Компоненты подистемы обработки прерываний. Таблица IDT, Сегмент TSS. Организация выбора обработчика прерывания.
        Маскирование прерываний, немаскируемые прерывания. Обработка прерывания на своём стеке и со сменой стека. Типичные пролог и эпилог функции
        обработчика прерывания. Инструкции: \textbf{int} и \textbf{iret}.
\end{enumerate}

\end{document}
